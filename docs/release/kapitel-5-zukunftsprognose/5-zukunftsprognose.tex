\chapter{Zukunftsprognose}\label{zukunftsprognose}

Die Standards der Web Component Technologien sind noch nicht
fertiggestellt und befinden sich momentan noch in Entwicklung. Um
allgemeingültige Standards für alle Browserhersteller zu gewährleisten,
werden sich diese vermutlich noch verändern. Jedoch werden sie dadurch
ein breites Spektrum an Zustimmung und Implementierung bekommen, da sie
eine zentrale Rolle spielen, wenn darum geht auch das Web zu einer
Plattform zu machen, für welche die Paradigmen von höheren
Programmiersprachen gelten. So soll es auch im Web möglich sein
Applikationen aus mehreren verschiedenen Komponenten, die gekapselt,
wartbar und interoperabel sind, zu entwickeln. Polymer wird dabei
weiterhin eine wichtige Rolle innehaben, da die Bibliothek das Arbeiten
mit den Web Components vereinfacht und zusätzliche Funktionalitäten
dafür leistet. Dabei könnte sogar ein Vergleich der Polymer Bibliothek
und der jQuery Bibliothek vorstellbar sein. Indem jQuery wurde zu einem
de facto Standard für das Arbeiten mit DOM Elementen, so könnte Polymer
ein Standard für das Arbeiten mit Web Components werden. Da Google
maßgeblich die Entwicklung der Standards der Web Component Technologien
vorantreibt und sich die anderen Browserhersteller teilweise daran
orientieren, könnte es sogar sein, dass zumindest Teile von Polymer zum
offiziellen Standard der Web Technologien wird. Wenn die Standards von
allen Browsern akzeptiert und implementiert wurden, sinkt auch die
Komplexität von Polymer um ein Vielfaches, da die komplette Schicht der
Polyfills wegfällt. Neben den Polyfills kann Polymer dann auf die
Nutzung des Bibliothek-internen Shady DOM verzichten und stattdessen mit
einer standardisierten, schnellen und leichtgewichtigen Version des
Shadow DOM arbeiten. Dadurch wird Polymer um einiges schneller arbeiten
und auch auf mobilen Geräten effizienten Einzug erhalten. Neben Polymer
basiert auch Angular ab Version 2.0 auf den Web Component Standards und
dessen Technologien benutzen, jedoch verfolgt Angular einen anderen
Ansatz als Polymer. Angular implementiert hierfür eigene Schicht, welche
auf die nativen Technologien zugreift und das Framework komplexer macht.
Da in Zukunft die Polyfills und der Shady DOM der Polymer Bibliothek
wegfallen, könnte Angular Polymer in sich integrieren, statt die Web
Components Standards selbst zu implementieren. Hierfür würde
beispielsweise die Micro Schicht in Frage kommen, da diese nur die
Grundfunktionalitäten für den Umgang mit den Web Components leistet.
Jedoch werden beide Bibliotheken weiterhin koexistieren und weder
Polymer Angular ersetzen oder andersrum, da Polymer nur eine
erweiterbare Bibliothek und Angular ein vollständiges Framework ist.
Beide Plattformen verfolgen daher unterschiedliche Ansätze und können
unterschiedliche Probleme lösen. Jedoch kann durch die Entwicklung der
Carbon Elemente die Polymer Bibliothek sukzessive als Framework
erweitert werden, da diese eine neue Möglichkeit bieten, wie
Applikationen strukturiert werden können. Durch sie können komplexe
Applikationen realisiert werden, da die Elemente sehr nah an der
Plattform selbst sind, was einige Vorteile mit sich bringt. So haben sie
wenige Abhängigkeiten, eine bessere Performanz, können leicht
ausgetauscht werden und bieten ein vereinfachtes Debugging an, da sie
die Tools der Plattform, statt die eines Frameworks benutzen. Polymer
kann also als eine Bibliothek mit einem optionalen Framework Plugin
verstanden werden, wobei das Framework nur die Meinung Googles'
widerspiegelt wie ein Framework funktionieren sollte. Dieses bietet, je
nach Anforderungen, Vor- und Nachteile wie jedes Andere Framework auch.
Es liegt somit weiterhin im Ermessen der Entwickler ob Angular oder
Polymer eingesetzt werden soll, oder ob sogar, bis zu einem gewissen
Grad, beides verwendet werden soll. Jedoch beschränkt sich Polymer nicht
nur auf den Einsatz in Angular, sondern kann mit jedem beliebigem
Framework kombiniert werden. Wie genau die Entwicklung von Polymer
weiter verläuft und ob sich diese auf Angular auswirken wird, liegt
jedoch ganz im Ermessen von Google.
