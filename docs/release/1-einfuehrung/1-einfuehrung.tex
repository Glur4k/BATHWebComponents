\section{Einführung}\label{einfuxfchrung}

Der Begriff ``Web Components'' ist ein Dachbegriff für mehrere
entstehende Standards {[}SelfHTML 2014{]}, die es für Webentwickler
ermöglichen sollen, komplexe Anwendungsentwicklungen mit einer neuen
Sammlung an Werkzeugen zu vereinfachen. Diese sollen die Wartbarkeit,
Interoperabilität und Kapselung verbessern und somit ein Plugin-System
für das Web schaffen. Durch die neuen Standards soll das Web zu einer
Plattform werden, die es ermöglicht die Web-Sprache selbst, \ac{HTML},
zu erweitern. Dies ist bisher nicht möglich, da die HTML-Technologie,
und somit die Möglichkeiten HTML-Tags zu benutzen, vom \ac{W3C}
definiert und standardisiert wird. Unter den wichtigsten der neuen
Standard sind die folgenden vier Technologien aufzuführen: Custom
Elements, Shadow DOM, HTML Templates und HTML Imports. Custom Elements
ermöglichen es einem Webentwickler eigene HTML-Tags und deren Verhalten
zu definieren, oder bereits vorhandene oder native HTML-Tags zu
erweitern. Das Shadow DOM (Document Object Model) stellt einen Sub-DOM
in einem HTML Element bereit, welcher dem Element zugehöriges Markup,
CSS und JavaScript kapselt. HTML Templates stellen, wie der Name
impliziert, einen Template-Mechanismus für HTML bereit {[}Peter Kröner
2014{]}. HTML Imports erlauben das Laden von HTML-Dokumenten in andere
HTML-Dokumente {[}Peter Kröner 2 2014{]}.

Diese neuen Technologien werden allerdings noch nicht von den
populärsten Browsern, zu welchen Google Chrome, Mozilla Firefox, Opera
und der Internet Explorer beziehungsweise Edge gehören, unterstützt. Des
Weiteren ist das Implementieren einer Applikation, welche diese
Technologien nativ benutzt, bisher sehr komplex und schwierig zu
organisieren. Im Zuge dessen, entwickelt Google aktiv an einer
Bibliothek namens Polymer, welche sich diesen Problemen annimmt.

Polymer stellt eine Sammlung von Polyfills (webcomponents.js) bereit,
die älteren Browsern die nötigen Features für den Einsatz von Web
Components beibringen. Ebenso soll das Erstellen von eigenen
HTML-Elementen mit der Polymer-Bibliothek, und der damit
bereitgestellten API, für Entwickler komfortabel gemacht werden. Um nun
bereits entwickelte Web Components einfach wiederverwenden zu können,
stellt Polymer eine Sammlung von fertigen Polymer-Elementen bereit.

Im Rahmen dieser Bachelorarbeit sollen die verschiedenen Technologien
unter dem Dachbegriff Web Components, sowie deren Funktionsweise ohne,
als auch mit der Polymer Bibliothek, untersucht werden. Zur
Veranschaulichung soll eine Web Komponente mit Hilfe von Polymer
implementiert werden und mit einer ähnlichen Implementierung mit
AngularJS verglichen werden. Am Beispiel einer Web Komponente in Form
einer Multi-Navigations-Applikation sollen die Vor- und Nachteile des
Einsatzes von Polymer in Hinblick auf Implementierung und Performanz
dargestellt werden.

\subsection{Quellen}\label{quellen}

\begin{itemize}
\item
  {[}Peter Kröner 2014{]} Peter Kröner, Web Components erklärt - Teil 1:
  Was sind Web Components?, 2014,
  /http://www.peterkroener.de/web-components-erklaert-teil-1-was-sind-web-components/
\item
  {[}Peter Kröner 2 2014{]} Peter Kröner, Das Web der Zukunft, 2014,
  http://webkrauts.de/artikel/2014/das-web-der-zukunft
\item
  {[}SelfHTML 2014{]} SelfHTML, Web Components - eine Einführung, 2014,
  https://blog.selfhtml.org/2014/12/09/web-components-eine-einfuehrung/
\end{itemize}
