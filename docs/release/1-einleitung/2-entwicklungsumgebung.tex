\section{Entwicklungsumgebung}\label{entwicklungsumgebung}

-- TODO: complete

Im Folgenden wird die für die Abschlussarbeit zugrunde liegende
Entwicklungsumgebung und die dafür verwendeten Programme erläutert.
Wichtiger Bestandteil dieser Abschlussarbeit sind kleine Code-Fragmente,
welche verschiedene Implementierungen zeigen sollen. Für die Entwicklung
dieser Code-Fragmente wird Sublime Text 3 verwendet, für die
grundliegende Ausführung aller Code-Fragmente wird Google Chrome in der
aktuellsten Version (46) benutzt. Jegliche dargestellten
Implementierungen in dieser Abschlussarbeit werden von diesem Browser
unterstützt.

\begin{itemize}
\tightlist
\item
  node.js
\item
  grunt
\item
  bower
\item
  yeoman
\item
  webapp
\item
  generator-element
\item
  generator-polymer
\end{itemize}

siehe tut: https://www.youtube.com/watch?v=ATPXN1-AWs8

Zur Hilfe genommen wird ebenfalls GIT und ein öffentliches Repository
auf Github (https://github.com/Glur4k/BATHWebComponents).

Eingesetzte Tools: - Sublime Text - Git - Google Chrome (v46) - Firefox
(v41.0.2)
