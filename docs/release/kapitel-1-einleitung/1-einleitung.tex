\chapter{Einleitung}\label{einleitung}

Der Begriff ``Web Components'' ist ein Dachbegriff für mehrere
entstehende Standards \cite{citeulike:13844988}, die es für
Webentwickler ermöglichen sollen, komplexe Anwendungsentwicklungen mit
einer neuen Sammlung an Werkzeugen zu vereinfachen. Diese sollen die
Wartbarkeit, Interoperabilität und Kapselung verbessern und somit ein
Plugin-System für das Web schaffen. Durch die neuen Standards soll das
Web zu einer Plattform werden, die es ermöglicht die Web-Sprache selbst,
HTML, zu erweitern. Dies ist bisher nicht möglich, da die
HTML-Technologie, und somit die Möglichkeiten HTML-Tags zu benutzen, vom
W3C definiert und standardisiert wird. Unter den wichtigsten der neuen
Standard sind die folgenden vier Technologien aufzuführen: Custom
Elements, Shadow DOM, HTML Templates und HTML Imports. Custom Elements
ermöglichen es einem Webentwickler eigene HTML-Tags und deren Verhalten
zu definieren, oder bereits vorhandene oder native HTML-Tags zu
erweitern. Das Shadow DOM stellt einen Sub-DOM in einem HTML Element
bereit, welcher dem Element zugehöriges Markup, CSS und JavaScript
kapselt. HTML Templates stellen, wie der Name impliziert, einen
Template-Mechanismus für HTML bereit \cite{citeulike:13842702} und
erlauben das Laden von HTML-Dokumenten in andere HTML-Dokumente
\cite{citeulike:13842701}.

Diese neuen Technologien werden allerdings noch nicht von den
populärsten Browsern, zu welchen Google Chrome, Mozilla Firefox, Opera
und der Internet Explorer, bzw. Edge, gehören, unterstützt. Des Weiteren
ist das Implementieren einer Applikation, welche diese Technologien
nativ benutzt, bisher sehr komplex und schwierig zu organisieren. Im
Zuge dessen, entwickelt Google aktiv an einer Bibliothek namens Polymer,
welche sich diesen Problemen annimmt. Polymer stellt dabei eine Reihe an
unterschiedlichen Schichten dar, welche den Umgang mit Web Components
vereinfachen sollen. So stellt Polymer eine Sammlung an Mechanismen
bereit, die älteren Browsern die nötigen Features für den Einsatz von
Web Components beibringen. Ebenso soll das Erstellen von eigenen
HTML-Elementen mit der Polymer-Bibliothek, und der damit
bereitgestellten API, für Entwickler komfortabel gemacht werden. Um nun
bereits entwickelte Web Components einfach wiederverwenden zu können,
bietet Polymer eine Sammlung von fertigen Polymer-Elementen an.

Web Components und die Polymer Bibliothek greifen stark in den
Entwicklungsprozess von Webseiten ein und sollen diesen Verbessern und
vereinfachen. Die Seitenbau GmbH interessiert sich stark für diese neue
Technologie, da Wiederverwendbarkeit, Wartbarkeit und neue Technologien
im Fokus des Frontend Engineerings des Unternehmens stehen. Die
Seitenbau GmbH ist ein mittelständischer IT-Dienstleister und
unterstützt seit 1996 Organisationen aus Privatwirtschaft und
öffentlicher Verwaltung bei der Planung, Konzeption und Umsetzung
hochwertiger Softwarelösungen für E-Business und E-Government. Zu den
Kernkompetenzen der Seitenbau GmbH zählen dabei vor allem das Frontend
Engineering \& Content Management, die Konzeption und Entwicklung von
Individualsoftware sowie der Aufbau von personalisierten Intranet- \&
Portallösungen.

Im Rahmen dieser Bachelorarbeit sollen die verschiedenen Technologien
unter dem Dachbegriff Web Components, sowie deren Funktionsweise ohne,
als auch mit der Polymer Bibliothek, untersucht werden. Zur
Veranschaulichung soll eine Web Component mit Hilfe von Polymer
implementiert, und mit einer ähnlichen Implementierung mit AngularJS
verglichen werden. Am Beispiel einer Web Component in Form einer
Multi-Navigations-Applikation sollen die Vor- und Nachteile des
Einsatzes von Polymer in Hinblick auf Implementierung und Performanz
dargestellt werden.
