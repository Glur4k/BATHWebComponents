

\begin{itemize}
\tightlist
\item
  \texttt{type}: Konstruktor-Typ der Property, er kann ein Boolean,
  Date, Number, String, Array oder Object sein.
\item
  \texttt{value}: Standardwert der Property, der Wert muss boolean,
  number, string oder eine Funktion sein.
\item
  \texttt{reflectToAttribute}: Boolean-Wert, gibt an ob die Property mit
  dem HTML-Attribut synchronisiert werden soll. Das bedeutet, dass falls
  die Property geändert wird, das zugehörige HTML-Attribut des Elements
  geändert wird, was äquivalent zu
  \texttt{this.setAttribute(property,\ value);} ist. Der Name des
  HTML-Attributs muss dabei in dem Properties-Objekt kleingeschrieben
  werden. Enthält der Name zusätzlich Bindestriche, so muss die Property
  kleingeschrieben und die Bindestriche entfernt werden, wobei jedes
  Wort nach einem Bindestrich groß geschrieben werden muss.
\item
  \texttt{readOnly}: Boolean-Wert, gibt an ob die Property nur gelesen
  oder auch geschrieben werden soll (siehe Abschnitt 5.1).
\item
  \texttt{notify}: Boolean-Wert, die Property ist für
  Two-Way-Data-Binding (siehe Abschnitt 5.1) verfügbar, falls true
  gesetzt ist. Zusätzlich wird ein \texttt{property-name-changed}-Event
  ausgelöst, wenn sich die Property ändert.
\item
  \texttt{computed}: Name der Funktion als String, welche den Wert der
  Property berechnen soll (siehe Abschnitt 4.1.4).
\item
  \texttt{observer}: Name der Funktion als String, welche aufgerufen
  wird, wenn der Wert der Property geändert wird (siehe Abschnitt
  4.1.5).
\end{itemize}

Die genannten Parameter sind alle optional anzugeben. Wird keine der
Parameter definiert, so kann die Property direkt mit dem \texttt{type}
definiert werden. Soll also beispielsweise eine Property
\texttt{propertyName} als String und ohne Parameter angegeben werden, so
würde dies mit \texttt{properties:\ \{\ propertyName:\ String\ \}}
erreicht werden.

\subsection{Computed Properties}\label{computed-properties}

Polymer unterstützt des Weiteren zusammengesetzte, virtuelle Properties,
welche aus anderen Properties berechnet werden. Um dies zu erreichen,
muss die dafür verwendete Funktion im \texttt{properties}-Objekt mit
entsprechenden Parametern angegeben werden. Soll beispielsweise die
virtuelle Property \texttt{result} die Zusammensetzung der Properties a
und b darstellen, so wird sie als
\texttt{result:\ \{\ type:\ String,\ computed:\ computeResult(a,\ b)\ \}}
im \texttt{properties}-Objekt angegeben. Die entsprechende Funktion muss
dann als \texttt{computeResult:\ function(a,\ b)\ \{\ ...\ \}} im
Polymer-Prototyp definiert werden. Diese wird einmalig aufgerufen, wenn
sich eine der Eigenschaften a oder b ändert und wenn keine von beiden
undefiniert ist. Der von ihr zurückgegebene Wert wird anschließend in
der Variable \texttt{result} gespeichert.

\subsection{Property Oberserver}\label{property-oberserver}

Wird für eine Property der Parameter \texttt{observer} angegeben, so
wird sie auf Änderungen überwacht. Die angegebene Funktion, welcher als
optionale Argumente den neuen und den alten Wert übergeben werden, wird
somit aufgerufen, falls der Wert der Property geändert wird. Allerdings
kann in dem \texttt{properties}-Objekt jeweils nur eine Property von
einem Observer überwacht werden. Sollen mehrere Properties von demselben
Observer überwacht werden, kann das \texttt{observers}-Array des
Polymer-Prototyps verwendet werden. Soll beispielsweise die Funktion
\texttt{computeResult(a,\ b)} ausgeführt werden, sobald sich eine der
Properties \texttt{a} oder \texttt{b} ändert, so kann diese Funktion in
das Array übernommen werden. Jedoch wird die Funktion auch hier nur dann
ausgeführt, wenn keiner der Werte undefiniert ist. Des Weiteren wird bei
angegebenen Funktionen - im Gegensatz zu den im
\texttt{properties}-Objekt angegebenen Observern - nur der neue Wert
statt des neuen und des alten Werts übergeben. Mit dem
\texttt{observers}-Array ist es auch möglich, Sub-Properties oder Arrays
zu überwachen.

\subsection{Das hostAttributes-Objekt}\label{das-hostattributes-objekt}

Zusätzlich zu den Declared Properties können auch HTML-Element-Attribute
im Polymer-Prototyp definiert werden. Hierzu bietet Polymer das
\texttt{hostAttributes}-Objekt an. Die darin angegebenen
Schlüssel-Wert-Paare werden beim initialen Erstellen des Elements auf
dessen Attribute abgebildet. Das \texttt{hostAttributes}-Objekt kann
dabei alle HTML-Attribute bis auf das \texttt{class}-Attribut
definieren, darunter fallen beispielsweise die Attribute
\texttt{data-*}, \texttt{aria-*} oder \texttt{href}. Wird im
\texttt{hostAttributes}-Objekt beispielsweise das Attribut
\texttt{selected} mit \texttt{true} definiert, wird es bei einem
\texttt{my-element}-Element in Form von
\texttt{\textless{}my-element\ selected\textgreater{}Item\textless{}/my-element\textgreater{}}
ausgegeben. Wichtig ist hier die Serialisierung der
Schlüssel-Wert-Paare. Wird ein String, Dates oder Numbers als Wert
übergeben, so werden diese als String serialisiert, werden jedoch Arrays
oder Objekte übergeben, so werden diese mittels JSON.stringify
serialisiert. Boolean-Werte werden bei \texttt{false} entfernt und bei
\texttt{true} angezeigt. Um Daten in der anderen Richtung von einem
HTML-Element an das \texttt{hostAttributes}-Objekt zu propagieren, muss
auf eine alternative Syntax zugegriffen werden (siehe Abschnitt 5.1.3).

\subsection{Lifecycle-Callback-Funktionen}\label{lifecycle-callback-funktionen}

Die nativen Lifecycle-Callback-Funktionen (siehe Abschnitt 2.2.6) werden
ebenso von Polymer unterstützt. Diese können in dem Prototyp als
Attribut bei ihrem normalen Namen oder in verkürzter Form angegeben
werden, so heißt die \texttt{createdCallback}-Methode \texttt{created},
die \texttt{attachedCallback}-Methode heißt \texttt{attached} etc.. Soll
beispielsweise die \texttt{created}-Methode definiert werden, so kann
diese mit \texttt{created:\ function\ \{\ ...\ \}} in dem Prototyp
angegeben werden. Zusätzlich bietet Polymer einen
\texttt{readyCallback}, welcher aufgerufen wird, nachdem Polymer das
Element erstellt und den lokalen DOM initialisiert hat, also nachdem
alle im lokalen DOM befindlichen Elemente konfiguriert wurden und
jeweils ihre \texttt{ready}-Methode aufgerufen haben. Sie ist besonders
hilfreich, wenn nach dem Laden der Komponente dessen DOM nachträglich
manipuliert werden soll. Falls mit den Lifecycle-Callbacks gearbeitet
wird, muss auf die richtige Anwendung der Reihenfolge geachtet werden.
So werden die Callbacks eines Elements in der Reihenfolge
\texttt{created}, \texttt{ready}, \texttt{factoryImpl} und
\texttt{attached} ausgeführt.

\section{Shadow DOM und HTML
Templates}\label{shadow-dom-und-html-templates}

Die bisher gezeigten Methoden ermöglichen das Erstellen einer
Polymer-Komponente, die jedoch noch kein internes Markup beinhaltet. Wie
bei den nativen Technologien können auch mit Polymer erzeugte Custom
Elements um HTML-Markup, den lokalen DOM, erweitert werden
{[}citeulike:13915080{]}. Hierzu dient das Polymer
\texttt{\textless{}dom-module\textgreater{}}-Element, welches als ID den
Wert der \texttt{is}-Property des Polymer-Prototyps haben muss. Der zu
verwendende HTML-Markup muss dann in einem
\texttt{\textless{}template\textgreater{}}-Tag dem
\texttt{\textless{}dom-module\textgreater{}} hinzugefügt werden. Auf das
Klonen des Inhalts des Templates mittels der
\texttt{importNode}-Funktion (siehe Abschnitt 2.4.3) kann hierbei
verzichtet werden, da Polymer diesen automatisch in den lokalen DOM des
Elements klont. Soll also der lokale DOM des
\texttt{\textless{}element-name\textgreater{}}-Tags deklariert werden,
so wird dies wie folgt erreicht:

\begin{Shaded}
\begin{Highlighting}[]
\KeywordTok{<dom-module}\OtherTok{ id=}\StringTok{"element-name"}\KeywordTok{>}
  \KeywordTok{<template>}\NormalTok{Local DOM / Inneres HTML Markup}\KeywordTok{</template>}

  \KeywordTok{<script>}
    \AttributeTok{Polymer}\NormalTok{(}\OperatorTok{\{}
      \DataTypeTok{is}\OperatorTok{:} \StringTok{'element-name'}
    \OperatorTok{\}}\NormalTok{)}\OperatorTok{;}
  \OperatorTok{<}\SpecialStringTok{/script>}
\SpecialStringTok{</dom}\OperatorTok{-}\NormalTok{module}\OperatorTok{>}
\end{Highlighting}
\end{Shaded}

Der deklarative Teil des Elements, das
\texttt{\textless{}dom-module\textgreater{}} und dessen Inhalte, sowie
der Imperative Teil mit dem \texttt{Polymer(\{\ ...\ \})}-Aufruf können
entweder in derselben oder in getrennten HTML-Dateien stehen. Hierbei
spielt es jedoch keine Rolle, ob das \texttt{\textless{}script}-Tag
innerhalb oder außerhalb des
\texttt{\textless{}dom-module\textgreater{}}-Tags steht, solange das
Template vor dem Polymer-Funktionsaufruf geparst wird.

\subsection{Shady DOM}\label{shady-dom}

Wie in Kapitel 2.3.9 gezeigt, wird der Shadow DOM nicht von allen
Browsern unterstützt, ebenso ist der Polyfill für diesen aufgrund dessen
schlechter Performanz (siehe Kapitel 2.7.4) nur als allerletzte Instanz
zu sehen. Aus diesen Gründen ist in Polymer der sogenannte ``Shady DOM''
implementiert {[}citeulike:13886251{]}. Dieser bietet einen dem Shadow
DOM ähnlichen Scope für den DOM-Tree. dabei rendert er den DOM, als wenn
kein Shadow DOM in dem Element vorhanden wäre. Dies bringt wiederum auch
die dadurch entstehenden Nachteile mit sich, wie dass internes Markup
nach außen sichtbar ist oder keine Shadow Boundary verfügbar ist. Der
Vorteil ist jedoch, dass der Shady DOM genug Methoden für seinen Scope
bereitstellt, um sich wie ein Shadow DOM verhalten zu können, ohne die
Performanz zu mindern. Hierzu ist es jedoch zwingend notwendig, die
eigens entwickelte Shady-DOM-API im Umgang mit dem DOM zu benutzen, was
mit der \texttt{Polymer.dom(node)}-Funktion erreicht wird. Will man
beispielsweise alle Kinder mit der \texttt{children}-Eigenschaft des
Shadow-Host-Elements \texttt{\textless{}my-element\textgreater{}}
selektieren, so erfolgt dies mit der Shady-DOM-API mittels
\texttt{var\ children\ =\ Polymer.dom(my-element).children;} statt mit
der normalen DOM-API mittels
\texttt{var\ children\ =\ \ document.getElementsByTagName("my-element"){[}0{]}.children;}.
Diese gibt dann nur die von außen übergebenen (Light-DOM-)Elemente
zurück, ohne die Elemente des internen Markups des Templates zu
berücksichtigen. Die Shady-DOM-API bildet dabei alle Funktionen der
nativen DOM-API ab und ist performanter als der Shadow-DOM-Polyfill, da
nicht dessen Verhalten, sondern nur ein eigener DOM-Scope implementiert
ist. Jedoch beschränkt sich Polymer nicht nur auf den eigenen Shady DOM,
vielmehr ist es mit dem nativen Shadow DOM kompatibel, sodass die
Shady-DOM-API auf das native Shadow DOM zugreifen kann, falls dieses von
dem Browser unterstützt wird. Dadurch kann eine Applikation
implementiert werden, die auf allen Plattformen mit einer verbesserten
Performanz ausgeführt wird, wovon besonders die mobilen Plattformen
profitieren. Standardmäßig benutzt Polymer jedoch immer die eigene
Shady-DOM-API, wie das verhindert werden kann, ist in Abschnitt 6.1.2
dargestellt.

\subsection{DOM-Knoten automatisch
finden}\label{dom-knoten-automatisch-finden}

Um das Traversieren in dem lokalen DOM zu beschleunigen, bietet Polymer
eine Hilfsfunktion für das automatische Finden eines Elements oder auch
``Automatic Node Finding'' an. Hierzu wird intern ein Mapping zu den
statisch erzeugten DOM-Elementen erzeugt indem jedes Element des lokalen
DOM-Templates, für welches eine ID definiert wurde, in dem
\texttt{this.\$}-Hash gespeichert wird. Hat nun also das Element
\texttt{\textless{}div\ id="wrapper"\textgreater{}\textless{}/div\textgreater{}}
in dem Template die ID \texttt{wrapper}, so kann es in Polymer mittels
\texttt{this.\$.wrapper} selektiert werden. Jedoch werden dem Hash nur
die statisch erzeugten DOM-Knoten hinzugefügt, dynamisch mittels
\texttt{dom-repeat} oder \texttt{dom-if} hinzugefügte Knoten allerdings
nicht. Die dynamisch hinzugefügten Knoten können mit der
\texttt{this.\$\$}-Funktion selektiert werden. So liefert die Funktion
\texttt{this.\$\$(selector);} das erste Element, welches von den im
Parameter \texttt{selector} enthaltenen CSS-Selektor selektiert wird.

\subsection{Content Projection}\label{content-projection}

Um nun Elemente des Light DOMs in das lokalen DOM der Komponente zu
injizieren, bietet Polymer ebenso das von den nativen Methoden bekannte
\texttt{\textless{}content\textgreater{}}-Element an, welches einen
Insertion Point des Light DOM im lokalen DOM der Komponente darstellt.
Wie auch bei den nativen Insertion Points kann das
\texttt{\textless{}content\textgreater{}}-Element auch nur selektierte
Inhalte injizieren, indem das \texttt{select}-Attribut mit einem
entsprechenden Selektor gesetzt wird. Falls der Shadow DOM in Polymer
verfügbar ist, so wird eine Zusammenstellung des Shadow DOM und dem
Light DOM gerendert. Ist der Shadow DOM jedoch nicht verfügbar und der
Shady DOM wird verwendet, so ist der zusammengesetzte DOM der
tatsächliche DOM des Elements. Auch kann in Polymer mittels der
\texttt{\_observer}-Eigenschaft überwacht werden, ob Kind-Elemente der
Komponente hinzugefügt oder von ihr entfernt werden. Dazu wird dieser
die Funktion \texttt{observeNodes(callback)} zugewiesen, welche
ausgeführt wird, wenn Elemente hinzugefügt oder entfernt werden. Der
Parameter \texttt{callback} ist dabei eine anonyme Funktion, welche als
Übergabewert das Objekt \texttt{info} hat, in welchem die hinzugefügten
oder entfernten Knoten enthalten sind. Eine Implementierung der
Überwachung des \texttt{contentNode}-Knotens auf Änderungen könnte dabei
wie folgt aussehen:

\begin{Shaded}
\begin{Highlighting}[]
\KeywordTok{this}\NormalTok{.}\AttributeTok{_observer} \OperatorTok{=} \VariableTok{Polymer}\NormalTok{.}\AttributeTok{dom}\NormalTok{(}\KeywordTok{this}\NormalTok{.}\VariableTok{$}\NormalTok{.}\AttributeTok{contentNode}\NormalTok{).}\AttributeTok{observeNodes}\NormalTok{(}\KeywordTok{function}\NormalTok{(info) }\OperatorTok{\{}
  \KeywordTok{this}\NormalTok{.}\AttributeTok{processNewNodes}\NormalTok{(}\VariableTok{info}\NormalTok{.}\AttributeTok{addedNodes}\NormalTok{)}\OperatorTok{;}
  \KeywordTok{this}\NormalTok{.}\AttributeTok{processRemovedNodes}\NormalTok{(}\VariableTok{info}\NormalTok{.}\AttributeTok{removedNodes}\NormalTok{)}\OperatorTok{;}
\OperatorTok{\}}\NormalTok{)}\OperatorTok{;}
\end{Highlighting}
\end{Shaded}

\subsection{CSS-Styling}\label{css-styling}

Die in Abschnitt 2.3.5 gezeigten Regeln für das Stylen des Shadow DOM
sind auch unter Polymer und dessen Shady DOM gültig
{[}citeulike:13915080{]}. Zusätzlich können Komponenten CSS-Properties
(also Variablen) nach außen sichtbar machen, damit diese von außerhalb
der Komponente gesetzt werden können, um somit das CSS in einer
gekapselten Komponente zu bestimmen. Hierbei können auch Standardangaben
gemacht werden, die von der Komponente übernommen werden, wenn die
Variable nicht definiert wird. Um eine Variable bereitzustellen, muss
diese der entsprechenden Eigenschaft mit der Syntax
\texttt{var(-\/-variable-name,\ default)} in den Style-Regeln der
Komponente angegeben werden. Das folgende Beispiel zeigt den DOM eines
\texttt{x-element}, welches die CSS-Variable
\texttt{x-element-button-color} und dem zugewiesenen Standardwert
\texttt{red} für einen Button in einem
\texttt{\textless{}div\textgreater{}}-Tag mit der Klasse
\texttt{x-element-container} bereitstellt.

x-element Komponente

\begin{Shaded}
\begin{Highlighting}[]
\KeywordTok{<dom-module}\OtherTok{ id=}\StringTok{"x-element"}\KeywordTok{>}
  \KeywordTok{<template>}
    \KeywordTok{<style>}
      \FloatTok{.x-element-container} \NormalTok{> button }\KeywordTok{\{}
        \KeywordTok{color:} \NormalTok{var(--x-element-button-color, }\DataTypeTok{red}\NormalTok{)}\KeywordTok{;}
      \KeywordTok{\}}
    \KeywordTok{</style>}
    \KeywordTok{<div}\OtherTok{ class=}\StringTok{"x-element-container"}\KeywordTok{>}
      \KeywordTok{<button>}\NormalTok{Ich bin ein Button}\KeywordTok{</button>}
    \KeywordTok{</div>}
  \KeywordTok{</template>}
\KeywordTok{</dom-module>}
\end{Highlighting}
\end{Shaded}

Die Applikation, welche die Komponente benutzt, kann nun die Variable
\texttt{-\/-x-element-button-color} definieren, wie nachfolgend gezeigt
wird.

\begin{Shaded}
\begin{Highlighting}[]
\KeywordTok{<style}\OtherTok{ is=}\StringTok{"custom-style"}\KeywordTok{>}
  \NormalTok{x-element }\KeywordTok{\{}
    \KeywordTok{--x-element-button-color:} \DataTypeTok{green}\KeywordTok{;}
  \KeywordTok{\}}
\KeywordTok{</style>}
\end{Highlighting}
\end{Shaded}

Das Attribut \texttt{is="custom-style"} des
\texttt{\textless{}style\textgreater{}}-Tag dient dabei als Anweisung
für den Polyfill, da CSS-Properties noch nicht von allen Browsern
unterstützt werden. Nun soll jedoch nicht für jedes CSS-Attribut eine
Variable angelegt werden, da dies schnell unübersichtlich werden kann.
Um mehrere CSS-Attribute einer Komponente ändern zu können, können
sogenannte Mixins erstellt werden. Diese sind eine Sammlung an Styles,
die auf eine Komponente angewendet werden können. Sie werden wie eine
CSS-Variable definiert, mit dem Unterschied, dass der Wert ein Objekt
ist welches ein oder mehrere Regeln definiert. Um ein Mixin nach außen
bereitzustellen, muss es in der Komponente in den CSS-Regeln mit
\texttt{@apply(-\/-mixin-name)} bereitgestellt werden. Es kann dann von
der Applikation verwendet werden. So kann das obige Beispiel um das
\texttt{-\/-x-element}-Mixin erweitert werden:

\begin{Shaded}
\begin{Highlighting}[]
\KeywordTok{<dom-module}\OtherTok{ id=}\StringTok{"x-element"}\KeywordTok{>}
  \KeywordTok{<template>}
    \KeywordTok{<style>}
      \FloatTok{.x-element-container} \NormalTok{> button }\KeywordTok{\{}
        \KeywordTok{color:} \NormalTok{var(--x-element-button-color, }\DataTypeTok{red}\NormalTok{)}\KeywordTok{;}
        \ErrorTok{@apply(--x-element);}
      \KeywordTok{\}}
    \KeywordTok{</style>}
  \KeywordTok{</template>}
\KeywordTok{</dom-module>}
\end{Highlighting}
\end{Shaded}

Wobei es wie folgt von der einsetzenden Applikation verwendet werden
könnte.

\begin{Shaded}
\begin{Highlighting}[]
\KeywordTok{<style}\OtherTok{ is=}\StringTok{"custom-style"}\KeywordTok{>}
  \NormalTok{x-element }\KeywordTok{\{}
    \KeywordTok{--x-element-button-color:} \DataTypeTok{green}\KeywordTok{;}
    \KeywordTok{--x-element:} \NormalTok{\{}
        \NormalTok{padding: }\DataTypeTok{10px}\KeywordTok{;}
        \KeywordTok{margin:} \DataTypeTok{10px}\KeywordTok{;}
    \KeywordTok{\}}\NormalTok{;}
  \NormalTok{\}}
\KeywordTok{</style>}
\end{Highlighting}
\end{Shaded}

\section{Gemeinsame Styles mehrerer
Komponenten}\label{gemeinsame-styles-mehrerer-komponenten}

Um nun Style-Regeln auf mehrere Komponenten anzuwenden, stellt Polymer
die sogenannten ``Style Modules'' bereit. Diese ersetzen die ab Version
1.1 nicht mehr unterstützte Möglichkeit, externe Stylesheets zu
verwenden. Style Modules sind dabei nichts anderes als Komponenten,
welche von allen Komponenten importiert werden können, die diese Styles
anwenden sollen.

\subsection{Style Module anlegen}\label{style-module-anlegen}

Um eine Komponente mit geteilten Style-Regeln zu erstellen, genügt es,
dass diese die Style-Regeln in einem
\texttt{\textless{}dom-module\textgreater{}}-Tag mit einer beliebigen ID
definiert. Dies wird in dem folgenden Beispiel der Datei
``shared-styles.html'' gezeigt. Darin wird eine Style-Regeln definiert,
die den Text aller Elemente mit der Klasse \texttt{wrapper} Rot anzeigen
lässt.

\begin{Shaded}
\begin{Highlighting}[]
\KeywordTok{<dom-module}\OtherTok{ id=}\StringTok{"shared-styles"}\KeywordTok{>}
  \KeywordTok{<template>}
    \KeywordTok{<style>}
      \FloatTok{.wrapper} \KeywordTok{\{} \KeywordTok{color:} \DataTypeTok{red}\KeywordTok{;} \KeywordTok{\}}
    \KeywordTok{</style>}
  \KeywordTok{</template>}
\KeywordTok{</dom-module>}
\end{Highlighting}
\end{Shaded}

\subsection{Style Module benutzen}\label{style-module-benutzen}

Damit eine Komponente diese Styles nutzen kann, muss sie sie zunächst
importieren und anschließend einen
\texttt{\textless{}style\textgreater{}}-Tag definieren, welcher als
\texttt{include}-Attribut den Namen der Komponente mit den geteilten
Style-Regeln hat. Die Styles werden darin importiert und auf die gesamte
Komponente angewendet. Somit wird der Text des
\texttt{\textless{}div\textgreater{}}-Tags mit der Klasse
\texttt{wrapper} rot dargestellt.

\begin{Shaded}
\begin{Highlighting}[]
\KeywordTok{<link}\OtherTok{ rel=}\StringTok{"import"}\OtherTok{ href=}\StringTok{"shared-styles.html"}\KeywordTok{>}
\KeywordTok{<dom-module}\OtherTok{ id=}\StringTok{"x-element"}\KeywordTok{>}
  \KeywordTok{<template>}
    \KeywordTok{<style}\OtherTok{ include=}\StringTok{"shared-styles"}\KeywordTok{></style>}
    \KeywordTok{<div}\OtherTok{ class=}\StringTok{"wrapper"}\KeywordTok{>}\NormalTok{Wrapper mit rotem Text}\KeywordTok{</div>}
  \KeywordTok{</template>}
  \KeywordTok{<script>}\AttributeTok{Polymer}\NormalTok{(}\OperatorTok{\{}\DataTypeTok{is}\OperatorTok{:} \StringTok{'x-element'}\OperatorTok{\}}\NormalTok{)}\OperatorTok{;<}\SpecialStringTok{/script>}
\SpecialStringTok{</dom}\OperatorTok{-}\NormalTok{module}\OperatorTok{>}
\end{Highlighting}
\end{Shaded}

\section{HTML Imports}\label{html-imports}

Um mehrere Polymer Komponenten oder Komponenten innerhalb anderer
Komponenten zu benutzen, verwendet Polymer HTML Imports. Diese
funktionieren analog zu der Verwendung mit der nativen HTML Imports
Technologie (siehe Abschnitt 2.5), wobei die selben Vor- und Nachteile
auftreten. Polymer kümmert sich dabei lediglich automatisch im
Hintergrund um die korrekte Einbindung der HTML-Dateien und dessen
Bereitstellung im Dokument, falls ein
\texttt{\textless{}link\ rel="import"\textgreater{}} in einer Komponente
enthalten ist. Das manuelle Einbinden mittels der speziellen
JavaScript-Methoden oder Eigenschaften, wie beispielsweise der
\texttt{import}-Eigenschaft des importierten Elements, wird somit
hinfällig.

\subsection{Dynamisches Nachladen von
HTML}\label{dynamisches-nachladen-von-html}

Falls HTML-Dateien dynamisch zur Laufzeit nachgeladen werden sollen,
bietet Polymer zusätzlich eine Hilfsfunktion an, mit der HTML Imports
nachträglich ausgeführt werden können {[}citeulike:13914840{]}. Die
hierfür bereitgestellte Funktion
\texttt{importHref(url,\ onload,\ onerror);} importiert beim Aufruf
dynamisch die angegebene HTML-Datei in das Dokument. Sie erstellt dabei
ein \texttt{\textless{}link\ rel="import"\textgreater{}}-Element mit der
angegeben URL und fügt es dem Dokument hinzu, sodass dieser ausgeführt
werden kann. Wenn der Link geladen wurde, also der
\texttt{onload}-Callback aufgerufen wird, ist die
\texttt{import}-Eigenschaft des Links der Inhalt des zurückgegebenen,
importierten HTML-Dokuments. Der Parameter \texttt{onerror} ist dabei
eine optionale Callback-Funktion, die beim Auftreten eines Fehlers
aufgerufen wird.
