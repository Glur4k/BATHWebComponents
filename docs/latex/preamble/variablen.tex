\newcommand{\thema}{Web-Component-basierte Entwicklung mit Polymer}
\newcommand{\schlagworte}{Web Components, Polymer, AngularJS, JavaScript, HTML, Custom Elements, HTML Templates, Shadow DOM, HTML Imports}
\newcommand{\zusammenfassung}{In der heutigen Webentwicklung kommt es häufig vor, dass für diverse Probleme oftmals die gleiche, oder eine ähnliche Lösung entwickelt werden muss, ohne dass es hierfür ein Plugin-System gibt. Diesem Problem widmen sich die Web Components, welche dieses bereitstellen. Der Begriff Web Components ist dabei ein Dachbegriff für mehrere entstehende Standards, welche jedoch noch nicht von allen Browsern unterstützt werden. Sie sollen es ermöglichen, wartbare, interoperable und gekapselte Komponenten zu entwickeln.

Die von Google entwickelte Library ``Polymer'' setzt diese Technologien um und soll den Umgang mit ihnen vereinfachen sowie sie auf ältere Browser portieren. Sie ermöglicht es, gekapselte Komponenten zu entwickeln, welche wiederum von Komponenten verwendet oder mit anderen Komponenten verbunden werden können. Dies ermöglicht die Realisierung komplexer Applikationen.

Mittels Beispielimplementierungen von Komponenten sowohl ohne, als auch mit Polymer sowie einer ähnlichen Implementierung mit dem Framework ``AngularJS'', werden die Unterschiede der jeweiligen Ansätze dargestellt.}
\newcommand{\ausgabedatum}{15.10.2015}
\newcommand{\abgabedatum}{15.02.2016}
\newcommand{\autor}{Sandro Tonon}
\newcommand{\autorStrasse}{Allemannenstra"se 10}
\newcommand{\autorPLZ}{78467}
\newcommand{\autorOrt}{Konstanz}
\newcommand{\autorGeburtsort}{Waldshut-Tiengen}
\newcommand{\autorGeburtsdatum}{02.07.1990}
\newcommand{\prueferA}{Prof. Dr. Marko Boger}
\newcommand{\prueferB}{Dipl. Inf. (FH) Andreas Maurer}
\newcommand{\firma}{Seitenbau GmbH}
\newcommand{\studiengang}{Angewandte Informatik}
