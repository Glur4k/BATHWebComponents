# Anhang

## HTML Struktur zum Benutzen einer Web Komponente mit den nativen APIs

**index.html**
```html
<!DOCTYPE html>
<html lang="en">
<head>
  <meta charset="UTF-8">
  <title>Demo of a Custom Element</title>

  <!-- Import the web component -->
  <link rel="import" href="elements/custom-element.html">
</head>
<body>

  <!-- Use the web component -->
  <custom-element theme="style1">Reader</custom-element>

</body>
</html>
```


## Implementierung einer Web Komponente mit den nativen APIs

**custom-element.html**
```html
<template id="myElementTemplate">
  <style>
    .outer { ... }
    .style1 { color: green; }
    .style2 { color: blue; }
    .name { font-size: 35pt; padding-top: 0.5em; }
  </style>

  <div class="outer">
    Welcome in the Web Component
    <div class="name">
      <content></content>
    </div>
  </div>
</template>

<script>
  // Refers to the "importee", which is index.html
  var importDoc = document.currentScript.ownerDocument;

  // Creates an object based in the HTML Element prototype
  var CustomElementProto = Object.create(HTMLElement.prototype);

  // Creates the "theme" attribute and sets a default value
  CustomElementProto.theme = 'style1';

  // Fires when an instance of the element is created
  CustomElementProto.createdCallback = function() {

    // Creates the shadow root
    var shadow = this.createShadowRoot();

    // Gets content from <template>
    var template = importDoc.querySelector('#myElementTemplate').content;

    // Adds a template clone into shadow root
    shadow.appendChild(template.cloneNode(true));

    // Caches .outer DOM query
    this.outer = shadow.querySelector('.outer');

    // Checks if the "theme" attribute has been overwritten
    if (this.hasAttribute('theme')) {
      var theme = this.getAttribute('theme');
      this.setTheme(theme);
    } else {
      this.setTheme(this.theme);
    }
  };

  // Fires when an attribute was added, removed, or updated
  CustomElementProto.attributeChangedCallback = function(attr, oldVal, newVal) {
    if (attr === 'theme') {
      this.setTheme(newVal);
    }
  };

  // Sets new value to "theme" attribute
  CustomElementProto.setTheme = function(val) {
    this.theme = val;
    this.outer.className = "outer " + this.theme;
  };

  // Registers <custom-element> in the main document
  document.registerElement("custom-element", {
    prototype: CustomElementProto
  });
</script>
```


## Polymer Seed-Element

```html
<!--
@license
Copyright (c) 2015 The Polymer Project Authors. All rights reserved.
This code may only be used under the BSD style license found at http://polymer.github.io/LICENSE.txt
The complete set of authors may be found at http://polymer.github.io/AUTHORS.txt
The complete set of contributors may be found at http://polymer.github.io/CONTRIBUTORS.txt
Code distributed by Google as part of the polymer project is also
subject to an additional IP rights grant found at http://polymer.github.io/PATENTS.txt
-->
<link rel="import" href="../polymer/polymer.html">

<!--
An element providing a solution to no problem in particular.

Example:

    <my-element></my-element>

@group Seed Elements
@element my-element
@demo demo/index.html
@hero hero.svg
-->
<dom-module id="my-element">

  <template>
    <style>
      :host {
        display: block;
        box-sizing: border-box;
      }

      .author img {
        display: block;
        float: left;
        margin-right: 5px;
        max-height: 100px;
        max-width: 100px;
      }
    </style>
    <h1>&lt;my-element&gt;</h1>
    <content></content>
    <p class="author">
      <img src="{{author.image}}">
      Cheers,<br/>
      <span class="name">{{author.name}}</span>
    </p>
  </template>

</dom-module>

<script>

  Polymer({

    is: 'my-element',

    properties: {

      /**
       * `fancy` indicates that the element should don a monocle and tophat,
       * while checking its pocket watch.
       */
      fancy: Boolean,

      /**
       * Describes the author of the element, but is really just an excuse to
       * show off JSDoc annotations.
       *
       * @type {{name: string, image: string}}
       */
      author: {
        type: Object,
        // Use `value` to provide a default value for a property, by setting it
        // on your element's prototype.
        //
        // If you provide a function, as we do here, Polymer will call that
        // _per element instance_.
        //
        // We do that to ensure that each element gets its own copy of the
        // value, rather than having it shared across all instances (via the
        // prototype).
        value: function() {
          return {
            name:  'Dimitri Glazkov',
            image: 'http://addyosmani.com/blog/wp-content/uploads/2013/04/unicorn.jpg',
          };
        }
      },

    },

    // Element Lifecycle

    ready: function() {
      // `ready` is called after all elements have been configured, but
      // propagates bottom-up. This element's children are ready, but parents
      // are not.
      //
      // This is the point where you should make modifications to the DOM (when
      // necessary), or kick off any processes the element wants to perform.
    },

    attached: function() {
      // `attached` fires once the element and its parents have been inserted
      // into a document.
      //
      // This is a good place to perform any work related to your element's
      // visual state or active behavior (measuring sizes, beginning animations,
      // loading resources, etc).
    },

    detached: function() {
      // The analog to `attached`, `detached` fires when the element has been
      // removed from a document.
      //
      // Use this to clean up anything you did in `attached`.
    },

    // Element Behavior

    /**
     * The `my-element-lasers` event is fired whenever `fireLasers` is called.
     *
     * @event my-element-lasers
     * @detail {{sound: String}}
     */

    /**
     * Sometimes it's just nice to say hi.
     *
     * @param {string} greeting A positive greeting.
     * @return {string} The full greeting.
     */
    sayHello: function(greeting) {
      var response = greeting || 'Hello World!';
      return 'my-element says, ' + response;
    },

    /**
     * Attempt to destroy this element's enemies with a beam of light!
     *
     * Or, at least, dispatch an event in the vain hope that someone else will
     * do the zapping.
     */
    fireLasers: function() {
      this.fire('my-element-lasers', {sound: 'Pew pew!'});
    }

  });

</script>
```


## Implementierung der Polymer Komponente Multi-Navigation-App

**multi-navigation-app.html**
```html
<!--
@license
Copyright (c) 2015 The Polymer Project Authors. All rights reserved.
This code may only be used under the BSD style license found at http://polymer.github.io/LICENSE.txt
The complete set of authors may be found at http://polymer.github.io/AUTHORS.txt
The complete set of contributors may be found at http://polymer.github.io/CONTRIBUTORS.txt
Code distributed by Google as part of the polymer project is also
subject to an additional IP rights grant found at http://polymer.github.io/PATENTS.txt
-->

<link rel="import" href="../paper-tabs/paper-tabs.html">
<link rel="import" href="../paper-toolbar/paper-toolbar.html">
<link rel="import" href="../iron-pages/iron-pages.html">
<link rel="import" href="../paper-drawer-panel/paper-drawer-panel.html">
<link rel="import" href="../paper-button/paper-button.html">
<link rel="import" href="../paper-icon-button/paper-icon-button.html">
<link rel="import" href="../paper-card/paper-card.html">
<link rel="import" href="../iron-icons/iron-icons.html">

<dom-module id="multi-navigation-app">

  <template>
    <style>
      :host {
        display: block;
        box-sizing: border-box;
      }
      paper-tabs {
        color: #fff;
        height: 100%;
        margin: auto -16px;
      }
      #topnavi ::content paper-tab {
        min-width: 50px;
      }
      @media only all and (min-width: 600px) {
        #topnavi ::content paper-tab {
          min-width: 80px;
        }
      }
      #leftnavi ::content paper-button {
        margin: 0;
        width: 100%;
        text-align: left;
        color: #3f51b5;
        text-transform: none;
      }
      #leftnavi ::content paper-button.iron-selected {
        font-weight: bold;
      }
      iron-pages {
        position: relative;
      }
      #paperDrawerPanel [drawer]:not([style-scope]):not(.style-scope) {
        border-right: 1px solid #e0e0e0;
      }
      paper-drawer-panel {
        --paper-drawer-panel-main-container: {
          background-color: #f5f5f5;
        };
        --paper-drawer-panel-left-drawer-container: {
          border-right: 1px solid #e5e5e5;
        };
      }
      #contents ::content paper-card {
        margin: 0 10%;
        width: 80%;
      }
    </style>

    <paper-drawer-panel>
      <paper-header-panel drawer>
        <paper-toolbar></paper-toolbar>

        <!-- Left Navigation -->
        <iron-pages id="leftnavi">
          <content select="iron-selector"></content>
        </iron-pages>
      </paper-header-panel>

      <paper-header-panel main>
        <paper-toolbar>
          <paper-icon-button icon="menu" paper-drawer-toggle></paper-icon-button>

          <!-- Top Navigation -->
          <paper-tabs id="topnavi" selected="{{selectedTop}}">
            <content select="paper-tab"></content>
          </paper-tabs>
        </paper-toolbar>

        <!-- Main Content -->
        <iron-pages id="contents" selected="{{selectedContent}}">
          <content select=".main"></content>
        </iron-pages>
      </paper-header-panel>
    </paper-drawer-panel>

  </div>
  </template>

</dom-module>

<script>

  Polymer({

    is: 'multi-navigation-app',

    properties: {
      selectedTop: {
        type: Number,
        value: 0
      },
      selectedContent: {
        type: Number,
        value: 0
      }
    },

    // Element Behavior
    _countLinks: function() {
      var count = 0;
      this.$.leftnavi.getContentChildren().forEach(function (value, i) {
        var links = value.children;
        for (j = 0; j < links.length; j++) {
          count++;
          links[j].linkNum = (count - 1);
        }
      });
    },

    _getSelectedContent: function() {
      return this.$.leftnavi.getContentChildren()[this.$.topnavi.selected].selectedItem.linkNum;
    },

    _addSelectedHandler: function() {
      var topNaviItems = this.$.topnavi;
      var leftNaviItems = this.$.leftnavi;
      var contents = this.$.contents;
      var that = this;

      topNaviItems.addEventListener('iron-select', function() {
        // init first selected
        leftNaviItems.select(that.selectedTop);
        var leftSelected = leftNaviItems.getContentChildren()[that.selectedTop].selectedItem;
        if (typeof leftSelected == 'undefined') {
          leftNaviItems.getContentChildren()[that.selectedTop].select(0)
        }

        contents.select(that._getSelectedContent());
      });

      leftNaviItems.getContentChildren().forEach(function (value, i) {
        value.addEventListener('iron-select', function (e) {
          contents.select(that._getSelectedContent());
        });
      });
    },
  });

</script>
```

Diese Komponente kann auch auf GitHub unter \texttt{https://github.com/glur4k/multi-navigation-app} und in dem customelements.io Elemente Katalog unter \texttt{https://customelements.io/glur4k/multi-navigation-app/}. Sie kann ebenfalls mit Bower `bower install multi-navigation-app` heruntergeladen werden.
