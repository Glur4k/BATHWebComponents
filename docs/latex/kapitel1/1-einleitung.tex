\chapter{Einleitung}\label{einleitung}

Der Begriff ``Web Components'' ist ein Dachbegriff für mehrere entstehende Standards \cite{citeulike:13844988}, welche es für Webentwickler ermöglichen sollen, komplexe Anwendungsentwicklungen mit einer neuen Sammlung an Werkzeugen zu vereinfachen. Diese sollen die Wartbarkeit, Interoperabilität und Kapselung verbessern und somit ein Plugin-System für das Web schaffen. Durch die neuen Standards soll das Web zu einer Plattform werden, die es ermöglicht, die Web-Sprache \ac{HTML} zu erweitern. Dies ist bisher nicht möglich, da die \ac{HTML}-Technologie -- und somit die Möglichkeiten, \ac{HTML}-Tags zu benutzen -- vom \ac{W3C} definiert und standardisiert wird. Unter den wichtigsten der neuen Standards sind die folgenden vier Technologien aufzuführen: Custom Elements, Shadow \ac{DOM}, \ac{HTML} Templates und \ac{HTML} Imports. Custom Elements ermöglichen es einem Webentwickler, eigene \ac{HTML}-Tags und deren Verhalten zu definieren, oder bereits vorhandene oder native \ac{HTML}-Tags zu erweitern. Das Shadow \ac{DOM} stellt ein Sub-\ac{DOM} in einem \ac{HTML}-Element bereit, welches dem Element zugehöriges Markup, \ac{CSS} und JavaScript kapselt. \ac{HTML} Templates stellen, wie der Name impliziert, einen Template-Mechanismus für \ac{HTML} bereit und \ac{HTML} Imports erlauben das Laden von \ac{HTML}-Dokumenten in andere \ac{HTML}-Dokumente. \cite{citeulike:13842702}, \cite{citeulike:13842701}

Diese neuen Technologien werden allerdings noch nicht vollständig von allen populären Browsern, zu welchen Google Chrome, Mozilla Firefox, Opera und der Internet Explorer bzw. Edge, gehören, unterstützt. Des Weiteren ist das Implementieren einer Applikation, welche diese Technologien nativ benutzt, bisher sehr komplex und schwierig zu organisieren. Im Zuge dessen entwickelt Google aktiv an einer Library namens ``Polymer'', welche sich diesen Problemen annimmt. Polymer stellt dabei eine Reihe an unterschiedlichen Schichten dar, welche den Umgang mit Web Components vereinfachen sollen. So stellt Polymer eine Sammlung an Mechanismen bereit, welche älteren Browsern die nötigen Features für den Einsatz von Web Components beibringen. Ebenso soll das Erstellen von eigenen \ac{HTML}-Elementen mit der Polymer-Library und der damit bereitgestellten \ac{API} für Entwickler komfortabler gemacht werden. Um bereits entwickelte Web Components einfach wiederverwenden zu können, bietet Polymer eine Sammlung von vorgefertigten Elementen an.

Web Components und die Polymer-Library greifen stark in den Entwicklungsprozess von Webseiten ein und sollen diesen verbessern und vereinfachen. Die Seitenbau GmbH interessiert sich stark für diese neue Technologie, da Wiederverwendbarkeit, Wartbarkeit und neue Technologien im Fokus des Frontend-Engineerings des Unternehmens stehen.
Die Seitenbau GmbH ist ein mittelständischer IT-Dienstleister und unterstützt seit 1996 Organisationen aus Privatwirtschaft und öffentlicher Verwaltung bei der Planung, Konzeption und Umsetzung hochwertiger Softwarelösungen für E-Business und E-Government. Zu den Kernkompetenzen der Seitenbau GmbH zählen dabei vor allem das Frontend Engineering und Content Management, die Konzeption und Entwicklung von Individualsoftware sowie der Aufbau von personalisierten Intranet- und Portallösungen.

Im Rahmen dieser Bachelorarbeit sollen die verschiedenen Technologien unter dem Dachbegriff Web Components sowie deren Funktionsweise sowohl ohne als auch mit der Polymer-Library untersucht werden. Zur Veranschaulichung soll eine Web Component mit Hilfe von Polymer implementiert und mit einer ähnlichen Implementierung mit AngularJS verglichen werden. Am Beispiel einer Web Component in Form einer Multi-Navigations-Applikation sollen die Vor- und Nachteile des Einsatzes von Polymer in Hinblick auf Implementierung und Performance dargestellt werden.

In Kapitel \ref{web-components-nach-w3c} werden die Standards der Web Components beschrieben, auf welche die in Kapitel \ref{einfuehrung-in-polymer} beschriebene Library Polymer aufsetzt. Wie sie dies im Detail umsetzt wird in Kapitel \ref{analogie-zu-nativen-web-components} dargestellt. In Kapitel \ref{zusaetzliche-polymer-funktionalitaeten} werden zusätzliche Funktionalitäten dieser Library aufgezeigt und in Kapitel \ref{best-practices-beim-arbeiten-mit-polymer} werden einige Best Practices im Umgang mit ihr erklärt. Die in den Kapiteln \ref{einfuehrung-in-polymer} bis \ref{best-practices-beim-arbeiten-mit-polymer} gewonnenen Erkenntnisse werden in Kapitel \ref{komponenten-entwicklung} in einer Beispielimplementierung umgesetzt und mit einer ähnlichen Implementierung mit AngularJS verglichen. In Kapitel \ref{zukunftsprognose} wird abschließend eine Zukunftsprognose aufgestellt.
