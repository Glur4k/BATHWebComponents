\chapter{Web Components nach dem vorläufigen W3C-Standard}\label{web-components-nach-w3c}

In diesem Kapitel wird auf die Problemlösungen der Web Components nach den Vorstellungen des \ac{W3C} eingegangen. In Abschnitt \ref{custom-elements} wird die erste Technologie vorgestellt, die Custom Elements, Abschnitt \ref{html-templates} widmet sich den \ac{HTML} Templates, in Abschnitt \ref{shadow-dom} wird auf den Shadow \ac{DOM} eingegangen und Abschnitt \ref{html-imports} zeigt die letzte Technologie, die \ac{HTML} Imports. In Abschnitt \ref{polyfills-mit-webcomponents.js} werden die Polyfills erklärt, welche für die Technologien noch zwingend notwendig sind. Abschließend wird in Kapitel \ref{implementierung-einer-komponente-mit-den-nativen-web-component-apis} anhand der in diesem Kapitel erklärten Technologien eine exemplarische Komponente implementiert.


\section{Problemlösung}\label{problemloesung}

In der heutigen Webentwicklung kommt es häufig vor, dass oftmals für diverse Probleme die gleiche, oder eine ähnliche Lösung programmiert werden muss. So muss auf vielen Seiten ein Slider, eine Navigation oder eine andere Komponente, welche das gewünschte Feature beinhaltet, eingebunden werden. Diese unterscheiden sich stets leicht, bringen im Kern aber dennoch meist die selben Funktionen mit sich. Um diese Funktionen auf der Webseite verfügbar zu machen, sind eine Reihe an verschiedenen Technologien notwendig. Wenn die gewünschte Komponente bereits existiert, muss für das Einbinden dieser Komponente ein bestimmtes \ac{HTML}-Markup geschrieben werden. Damit die Komponente nun funktioniert, muss ein JavaScript eingebunden werden, welches zusätzlich noch anhand einer vordefinierten \ac{API} konfiguriert werden muss. Diese \ac{API} ist in der Regel nur für diese eine Komponente entworfen, so müssen für jede Komponente unterschiedliche \ac{API}s angesprochen werden, die sich mitunter stark unterscheiden können. Damit die Komponente dann auch in das visuelle Design der eigenen Webseite passt, muss ebenso ein entsprechendes Stylesheet mit den Style-Definitionen eingebunden werden. Da \ac{CSS}-Regeln immer global auf das gesamte Dokument angewendet werden, kann es dabei zu ungewollten Auswirkungen auf andere Bestandteile der Webseite kommen. In diesem Fall muss auch das Stylesheet noch nachgebessert werden. Nimmt man diese Punkte zusammen, so wird deutlich, dass es in der Webentwicklung kein Plugin-System gibt, mit dem Webseiten schnell und einfach erweitert werden können.

Diesem Problem widmen sich die Web Components. Sie sollen der Frontend-Entwicklung ein Plugin-System bereitstellen, welches es ermöglicht, fremde und eigene Komponenten schnell und einheitlich in die eigene Seite einzubinden. Eine Komponente steht dabei als eigenes \ac{HTML}-Element, welches ihre gesamte innere Funktionalität in sich kapselt und nach außen unsichtbar macht. Konflikte mit anderen Komponenten oder der einbindenden Webseite selbst werden somit vermieden. Dabei ist das Verhalten nach außen für jede Komponente dasselbe, es gibt also für jede Komponente die gleiche Schnittstelle, um sie zu konfigurieren und einzubinden. Dies erleichtert den Umgang mit Plugins deutlich, da die einzige dafür notwendige Technologie \ac{HTML} selbst ist. Dadurch können einzelne Komponenten verwendet werden wie jedes \ac{HTML}-Element. Sie sind verschachtelbar und haben Attribute, über welche sie konfiguriert werden können. Web Components bilden dabei eine Sammlung an Technologien, um jene Eigenschaften zu gewährleisten. In den folgenden Abschnitten werden die grundlegenden Technologien erklärt und auf ihre Anwendung eingegangen.
